\documentclass{beamer}
\usetheme{Copenhagen}
\setbeamertemplate{navigation symbols}{}
\usepackage{amsmath,amssymb,amsfonts,amsthm}
\usepackage{algorithmic}
\usepackage{graphicx}
\usepackage{textcomp}
\usepackage{txfonts}
\usepackage{listings}
\usepackage{mathtools}
\usepackage{gensymb}
\usepackage{hyperref}
\usepackage{optidef}
\usepackage{tkz-euclide} % loads  TikZ and tkz-base
\newcommand\myeq{\mathrel{\stackrel{\makebox[0pt]{\mbox{\normalfont\tiny iid}}}{\sim}}}
\usepackage{listings}
    \usepackage{color}                                            %%
    \usepackage{array}                                            %%
    \usepackage{longtable}                                        %%
    \usepackage{calc}                                             %%
    \usepackage{multirow}                                         %%
    \usepackage{hhline}                                           %%
    \usepackage{ifthen}    
    \usepackage{lscape}
%\usetheme{Frankfurt}
\providecommand{\abs}[1]{\left\vert#1\right\vert}
\title{Gate 2023 ST Q26}
\author{Mayank Gupta}
\institute{IIT Hyd}
\date{\today}

\begin{document}
\lstset{
    language=Python,   % Set the programming language for syntax highlighting
    basicstyle=\ttfamily, % Set the font style for the code
    keywordstyle=\color{blue}, % Customize keywords
    commentstyle=\color{green}, % Customize comments
    stringstyle=\color{red},   % Customize strings
    numbers=left,      % Display line numbers
    numberstyle=\tiny, % Set the style for line numbers
    breaklines=true,   % Automatically wrap long lines
}
\lstset{
    language=C,               % Set the programming language
    basicstyle=\ttfamily,     % Font style for the code
    keywordstyle=\color{blue}, % Keyword style
    commentstyle=\color{green},% Comment style
    stringstyle=\color{red},   % String style
    numbers=left,             % Display line numbers
    numberstyle=\tiny,        % Style for line numbers
    breaklines=true,           % Automatically wrap long lines
    frame=single,             % Add a frame around the code
    showstringspaces=false,    % Don't show spaces within strings
    tabsize=4,                % Set tab size to 4 spaces
    morekeywords={printf, scanf, int, main, if, else, while}, % Additional keywords
    extendedchars=true,       % Allow extended characters like underscores
    literate={~} {$\sim$}{1}, % Replace ~ with a tilde symbol
    backgroundcolor=\color{gray!10}, % Background color for the code
    escapeinside={(*@}{@*)},  % Define an escape sequence for LaTeX code within comments
}
\providecommand{\pr}[1]{\ensuremath{\Pr\left(#1\right)}}
\providecommand{\prt}[2]{\ensuremath{p_{#1}^{\left(#2\right)} }}        % own macro for this question
\providecommand{\qfunc}[1]{\ensuremath{Q\left(#1\right)}}
\providecommand{\sbrak}[1]{\ensuremath{{}\left[#1\right]}}
\providecommand{\lsbrak}[1]{\ensuremath{{}\left[#1\right.}}
\providecommand{\rsbrak}[1]{\ensuremath{{}\left.#1\right]}}
\providecommand{\brak}[1]{\ensuremath{\left(#1\right)}}
\providecommand{\lbrak}[1]{\ensuremath{\left(#1\right.}}
\providecommand{\rbrak}[1]{\ensuremath{\left.#1\right)}}
\providecommand{\cbrak}[1]{\ensuremath{\left\{#1\right\}}}
\providecommand{\lcbrak}[1]{\ensuremath{\left\{#1\right.}}
\providecommand{\rcbrak}[1]{\ensuremath{\left.#1\right\}}}
\newcommand{\sgn}{\mathop{\mathrm{sgn}}}
\providecommand{\abs}[1]{\left\vert#1\right\vert}
\providecommand{\res}[1]{\Res\displaylimits_{#1}} 
\providecommand{\norm}[1]{\left\lVert#1\right\rVert}
%\providecommand{\norm}[1]{\lVert#1\rVert}
\providecommand{\mtx}[1]{\mathbf{#1}}
\providecommand{\mean}[1]{E\left[ #1 \right]}
\providecommand{\cond}[2]{#1\middle|#2}
\providecommand{\fourier}{\overset{\mathcal{F}}{ \rightleftharpoons}}
\newenvironment{amatrix}[1]{%
  \left(\begin{array}{@{}*{#1}{c}|c@{}}
}{%
  \end{array}\right)
}
\newcommand{\cosec}{\,\text{cosec}\,}
\providecommand{\dec}[2]{\ensuremath{\overset{#1}{\underset{#2}{\gtrless}}}}
\newcommand{\myvec}[1]{\ensuremath{\begin{pmatrix}#1\end{pmatrix}}}
\newcommand{\mydet}[1]{\ensuremath{\begin{vmatrix}#1\end{vmatrix}}}
\newcommand{\myaugvec}[2]{\ensuremath{\begin{amatrix}{#1}#2\end{amatrix}}}
\providecommand{\rank}{\text{rank}}
\providecommand{\pr}[1]{\ensuremath{\Pr\left(#1\right)}}
\providecommand{\qfunc}[1]{\ensuremath{Q\left(#1\right)}}
	\newcommand*{\permcomb}[4][0mu]{{{}^{#3}\mkern#1#2_{#4}}}
\newcommand*{\perm}[1][-3mu]{\permcomb[#1]{P}}
\newcommand*{\comb}[1][-1mu]{\permcomb[#1]{C}}
\providecommand{\qfunc}[1]{\ensuremath{Q\left(#1\right)}}
\providecommand{\gauss}[2]{\mathcal{N}\ensuremath{\left(#1,#2\right)}}
\providecommand{\diff}[2]{\ensuremath{\frac{d{#1}}{d{#2}}}}
\providecommand{\myceil}[1]{\left \lceil #1 \right \rceil }
\newcommand\figref{Fig.~\ref}
\newcommand\tabref{Table~\ref}
\newcommand{\sinc}{\,\text{sinc}\,}
\newcommand{\rect}{\,\text{rect}\,}

\begin{frame}
  \titlepage
\end{frame}


\begin{frame}{Question}
Consider the following regression model
\begin{center}
	$y_{k} = \alpha_{0} + \alpha_{1} \log_{e}k + \epsilon_{k}, \qquad k = 1,2,…,n,$\\
\end{center}
where $\epsilon_{k}$'s are independent and identically distributed random variables each
having probability density function $ f\brak{x} = \frac{1}{2} e^{-|x|}, x \in \mathbb{R}$. Then which one of
the following statements is true? 
\begin{enumerate}
	\item The maximum likelihood estimator of $\alpha_{0}$ does not exist
	\item The maximum likelihood estimator of $\alpha_{1}$ does not exist
	\item The least squares estimator of $\alpha_{0}$ exists and is unique
	\item The least squares estimator of $\alpha_{1}$ exists, but it is not unique
\end{enumerate}
\end{frame}


\begin{frame}[allowframebreaks]{Likelihood Function}
%\begin{itemize}
\begin{align}
	f(\epsilon_{k}) &= \frac{1}{2} e^{-|\epsilon_{k}|}\\
	\text{Likelihood function}: f(\epsilon_{1}\epsilon_{2}....\epsilon_{n}) &= \prod_{k = 1}^{n} f(\epsilon_{k})\\
	L     &= \prod_{k = 1}^{n}\frac{1}{2}e^{-|\epsilon_{k}|}\\
L_{1} = \ln L &= \ln\brak{\prod_{k = 1}^{n}\frac{1}{2}e^{-|\epsilon_{k}|}}\\
	      &= \sum_{k = 1}^{n} \ln\brak{\frac{1}{2}e^{-|\epsilon_{k}|}}
\end{align}
%\end{frame}
%
%
%\begin{frame}
\begin{align}
	      &= \sum_{k = 1}^{n} \brak{-\ln 2 -|y_{k}-\alpha_{0}-\alpha_{1} \log_{e}k|}\\
	      &= -n\ln 2 -\sum_{k = 1}^{n} \brak{|y_{k}-\alpha_{0}-\alpha_{1} \log_{e}k|}\\  
	L_{1} &= \text{function of }\alpha_{0}, \alpha_{1}
\end{align}
We need to find the value of $\alpha_{0}$ and $\alpha_{1}$ which will maximise the value of $L_{1}$ i.e. the value of 
	$\alpha_{0}$ and $\alpha_{1}$ which will minimise the value of $\sum_{k = 1}^{n}|y_{k}-\alpha_{0}-\alpha_{1} \log_{e}k|$
\end{frame}


\begin{frame}[allowframebreaks]{Maximum Likelihood Estimator for $\alpha_{0}$}
\begin{enumerate}
	\item For $y_{k}-\alpha_{0}-\alpha_{1} \log_{e}k > 0$
\begin{mini*}|s|
{\alpha_{0}}{y_{k}-\alpha_{0}-\alpha_{1} \log_{e}k}
{}{}
\addConstraint{\alpha_{0}\leq y_{k}-\alpha_{1}\log_{e}k}{}
\end{mini*}
Using Lagrange multiplier method
\begin{align}
	L\brak{\lambda} &= y_{k}-\alpha_{0}-\alpha_{1} \log_{e}k - \lambda(\alpha_{0}-y_{k}+\alpha_{1} \log_{e}k)\\
	\frac{\partial L}{\partial \alpha_{0}} &= -1-\lambda = 0\\
	\frac{\partial L}{\partial \lambda} &= y_{k}-\alpha_{0}-\alpha_{1} \log_{e}k = 0\\
	\lambda &= -1\\
	\alpha_{0} &= y_{k} - \alpha_{1} \log_{e}k
\end{align}
%\end{enumerate}
%\end{frame}
%
%\begin{frame}
%\begin{enumerate}
\item For $y_{k}-\alpha_{0}-\alpha_{1} \log_{e}k < 0$
\begin{mini*}|s|
{\alpha_{0}}{-\brak{y_{k}-\alpha_{0}-\alpha_{1} \log_{e}k}}
{}{}
\addConstraint{\alpha_{0}\geq y_{k}-\alpha_{1}\log_{e}k}{}
\end{mini*}
Using Lagrange multiplier method
\begin{align}
	L\brak{\lambda} &= -\brak{y_{k}-\alpha_{0}-\alpha_{1} \log_{e}k} - \lambda(\alpha_{0}-y_{k}+\alpha_{1} \log_{e}k)\\
        \frac{\partial L}{\partial \alpha_{0}} &= 1-\lambda = 0\\
        \frac{\partial L}{\partial \lambda} &= y_{k}-\alpha_{0}-\alpha_{1} \log_{e}k = 0\\
        \lambda &= 1\\
        \alpha_{0} &= y_{k} - \alpha_{1} \log_{e}k
\end{align}
As value of $\alpha_{0}$ matches for both cases of modulus\\
$\therefore$ The maximum likelihood estimator of $\alpha_{0}$ exist
\end{enumerate}
\end{frame}


\begin{frame}[allowframebreaks]{Maximum Likelihood Estimator for $\alpha_{1}$}
\begin{enumerate}
	\item For $y_{k}-\alpha_{0}-\alpha_{1} \log_{e}k > 0$
\begin{mini*}|s|
{\alpha_{1}}{y_{k}-\alpha_{0}-\alpha_{1} \log_{e}k}
{}{}
\addConstraint{\alpha_{1}\leq \frac{y_{k}-\alpha_{0}}{\log_{e}k}}{}
\end{mini*}
Using Lagrange multiplier method
\begin{align}
	L\brak{\lambda} &= y_{k}-\alpha_{0}-\alpha_{1} \log_{e}k - \lambda \brak{\alpha_{1}-\frac{y_{k}-\alpha_{0}}{\log_{e}k}}\\
        \frac{\partial L}{\partial \alpha_{1}} &= -\log_{e}k -\lambda = 0\\
	\frac{\partial L}{\partial \lambda} &= -\brak{\alpha_{1}-\frac{y_{k}-\alpha_{0}}{\log_{e}k}} = 0\\
        \lambda &= -\log_{e}k\\
	\alpha_{1} &= \frac{y_{k} - \alpha_{0}}{\log_{e}k}
\end{align}
	\item For $y_{k}-\alpha_{0}-\alpha_{1} \log_{e}k < 0$
\begin{mini*}|s|
{\alpha_{1}}{-\brak{y_{k}-\alpha_{0}-\alpha_{1} \log_{e}k}}
{}{}
\addConstraint{\alpha_{1}\geq \frac{y_{k}-\alpha_{0}}{\log_{e}k}}{}
\end{mini*}
Using Lagrange multiplier method
\begin{align}
	L\brak{\lambda} &= -\brak{y_{k}-\alpha_{0}-\alpha_{1} \log_{e}k} - \lambda \brak{\alpha_{1}-\frac{y_{k}-\alpha_{0}}{\log_{e}k}}\\
        \frac{\partial L}{\partial \alpha_{1}} &= \log_{e}k -\lambda = 0\\
        \frac{\partial L}{\partial \lambda} &= -\brak{\alpha_{1}-\frac{y_{k}-\alpha_{0}}{\log_{e}k}} = 0\\
        \lambda &= \log_{e}k\\
        \alpha_{1} &= \frac{y_{k} - \alpha_{0}}{\log_{e}k}
\end{align}
As value of $\alpha_{1}$ matches for both cases of modulus\\
$\therefore$ The maximum likelihood estimator of $\alpha_{1}$ exist\\
$\therefore$ Option \brak{A} and \brak{B} are incorrect\\
\end{enumerate}
\end{frame}


\begin{frame}[allowframebreaks]{Least Square Estimator}
The least square estimator of $\alpha_{0}$ and $\alpha_{1}$ is $\tilde{\alpha_{0}}$ and $\tilde{\alpha_{1}}$ which will minimise
\begin{align}
Q\brak{\alpha_{0},\alpha_{1}} &=  \sum_{k = 1}^{n} \brak{y_{k}-\alpha_{0}-\alpha_{1} \log_{e}k}^2
\end{align}
\begin{table}[!htb]
	\input{./tables/table.tex}
	\caption{Variables used}
	\label{table_gate23_st_26}
\end{table}
\begin{align}
	%Q\brak{\alpha_{0},\alpha_{1}} &=  \sum_{k = 1}^{n} \brak{y_{k}-\alpha_{0}-\alpha_{1} \log_{e}k}^2\\
\frac{\partial Q}{\partial \alpha_{0}} &= -2 \sum_{k = 1}^{n} \brak{y_{k}-\alpha_{0}-\alpha_{1} \log_{e}k} = 0\\
	\sum_{k = 1}^{n} \brak{y_{k}-\alpha_{0}-\alpha_{1} \log_{e}k} &= 0\\
	n\bar{y} - n\alpha_{0} - \alpha_{1}n\bar{x} &= 0 \\%,\quad \text{where } \bar{x} = \frac{1}{n} \sum_{k = 1}^{n}\log_{e}k,\quad \bar{y} = \frac{1}{n} \sum_{k = 1}^{n}y_{k}\\
	\implies \tilde{\alpha_{0}} &= \bar{y} - \tilde{\alpha_{1}}\bar{x}\\
	\frac{\partial Q}{\partial \alpha_{1}} &= -2 \sum_{k = 1}^{n} \brak{y_{k}-\alpha_{0}-\alpha_{1} \log_{e}k}\log_{e}k = 0\\
	\implies \tilde{\alpha_{1}} &= \frac{\sum_{k = 1}^{n}\brak{\log_{e}k-\bar{x}}\brak{y_{k}-\bar{y}}}{\sum_{k = 1}^{n}\brak{\log_{e}k-\bar{x}}^2}
\end{align}
$\therefore$ Least square estimator of $\alpha_{0}$ and $\alpha_{1}$ exists and are unique\\
$\therefore$ Option \brak{C} is correct and \brak{D} is incorrect\\
\end{frame}


\begin{frame}[allowframebreaks]{Simulation}
probability density function is $ f\brak{x} = \frac{1}{2} e^{-|x|}$
\begin{enumerate}
\item Write a function cdf for calculating the cdf of any random variable\\
	 \begin{align}
  p_X(x) &= 
  \begin{cases}
          \frac{1}{2} e^{x} & x \le 0
  \\
          \frac{1}{2} e^{-x} &  x > 0
  \end{cases}
  \end{align}

  \begin{align}
  F_X(x) &= 
  \begin{cases}
	  \int_{-\infty}^{x} \brak{\frac{1}{2} e^{x}} dx  & x \le 0
  \\
	  \int_{-\infty}^{0} \brak{\frac{1}{2} e^{x}} dx + \int_{0}^{x} \brak{\frac{1}{2} e^{-x}} dx &  x > 0
  \end{cases}
  \end{align}

  \begin{align}
  F_X(x) &= 
  \begin{cases}
	  \frac{1}{2} e^{x} & x \le 0
  \\
	  \frac{1}{2} \brak{2-e^{-x}} &  x > 0
  \end{cases}
  \end{align}
\item Declare a function inverse cdf \brak{I\brak{u}} such that its input is any random number 
	and output is random variable whose cdf equals that of the given distribution\\
	For x $\le$ 0
		\begin{align}
			u &= \frac{1}{2} e^{x}\\
			e^{x} &= 2u\\
			x &= \ln{2u}\\
			\because x \le 0\\
			u \le 0.5
		\end{align}
	For x $>$ 0
		\begin{align}
			u &= \frac{1}{2} \brak{2-e^{-x}}\\
			2-e^{-x} &= 2u\\
			e^{-x} &= 2-2u\\
			x &= -\ln \brak{2-2u}\\
			\because x > 0\\
			u > 0.5
		\end{align}
  \begin{align}
  I\brak{u} &=
  \begin{cases}
	  \ln\brak{2u} & u \le 0.5
  \\
	  -\ln \brak{2-2u} &  u > 0.5
  \end{cases}
  \end{align}
\item Define three arrays random\textunderscore{vars} , cdf\textunderscore{values}
	, theoretical\textunderscore{cdf}\textunderscore{values}
	to store random variables, simulated cdf values and theoretical cdf values
\item Generate random numbers using rand() and calling inverse cdf funtion to generate our random variable
\item Calling cdf function to calculate the cdf of the generated random variable
\item Storing the random variable,theoretical cdf and generated cdf into their respective arrays
\item Storing the data of these three array into a .dat file
\item Plotting these .dat file in python
\end{enumerate}
\end{frame}
\end{document}

