\documentclass{beamer}
\usetheme{Copenhagen}
\setbeamertemplate{navigation symbols}{}
\usepackage{amsmath,amssymb,amsfonts,amsthm}
\usepackage{algorithmic}
\usepackage{graphicx}
\usepackage{textcomp}
\usepackage{txfonts}
\usepackage{listings}
\usepackage{mathtools}
\usepackage{gensymb}
\usepackage{hyperref}
\usepackage{optidef}
\usepackage{tkz-euclide} % loads  TikZ and tkz-base
\newcommand\myeq{\mathrel{\stackrel{\makebox[0pt]{\mbox{\normalfont\tiny iid}}}{\sim}}}
\usepackage{listings}
    \usepackage{color}                                            %%
    \usepackage{array}                                            %%
    \usepackage{longtable}                                        %%
    \usepackage{calc}                                             %%
    \usepackage{multirow}                                         %%
    \usepackage{hhline}                                           %%
    \usepackage{ifthen}    
    \usepackage{lscape}
%\usetheme{Frankfurt}
\providecommand{\abs}[1]{\left\vert#1\right\vert}
\title{Gate 2023 ST Q26}
\author{Mayank Gupta}
\institute{IIT Hyd}
\date{\today}

\begin{document}
\lstset{
    language=Python,   % Set the programming language for syntax highlighting
    basicstyle=\ttfamily, % Set the font style for the code
    keywordstyle=\color{blue}, % Customize keywords
    commentstyle=\color{green}, % Customize comments
    stringstyle=\color{red},   % Customize strings
    numbers=left,      % Display line numbers
    numberstyle=\tiny, % Set the style for line numbers
    breaklines=true,   % Automatically wrap long lines
}
\lstset{
    language=C,               % Set the programming language
    basicstyle=\ttfamily,     % Font style for the code
    keywordstyle=\color{blue}, % Keyword style
    commentstyle=\color{green},% Comment style
    stringstyle=\color{red},   % String style
    numbers=left,             % Display line numbers
    numberstyle=\tiny,        % Style for line numbers
    breaklines=true,           % Automatically wrap long lines
    frame=single,             % Add a frame around the code
    showstringspaces=false,    % Don't show spaces within strings
    tabsize=4,                % Set tab size to 4 spaces
    morekeywords={printf, scanf, int, main, if, else, while}, % Additional keywords
    extendedchars=true,       % Allow extended characters like underscores
    literate={~} {$\sim$}{1}, % Replace ~ with a tilde symbol
    backgroundcolor=\color{gray!10}, % Background color for the code
    escapeinside={(*@}{@*)},  % Define an escape sequence for LaTeX code within comments
}
\providecommand{\pr}[1]{\ensuremath{\Pr\left(#1\right)}}
\providecommand{\prt}[2]{\ensuremath{p_{#1}^{\left(#2\right)} }}        % own macro for this question
\providecommand{\qfunc}[1]{\ensuremath{Q\left(#1\right)}}
\providecommand{\sbrak}[1]{\ensuremath{{}\left[#1\right]}}
\providecommand{\lsbrak}[1]{\ensuremath{{}\left[#1\right.}}
\providecommand{\rsbrak}[1]{\ensuremath{{}\left.#1\right]}}
\providecommand{\brak}[1]{\ensuremath{\left(#1\right)}}
\providecommand{\lbrak}[1]{\ensuremath{\left(#1\right.}}
\providecommand{\rbrak}[1]{\ensuremath{\left.#1\right)}}
\providecommand{\cbrak}[1]{\ensuremath{\left\{#1\right\}}}
\providecommand{\lcbrak}[1]{\ensuremath{\left\{#1\right.}}
\providecommand{\rcbrak}[1]{\ensuremath{\left.#1\right\}}}
\newcommand{\sgn}{\mathop{\mathrm{sgn}}}
\providecommand{\abs}[1]{\left\vert#1\right\vert}
\providecommand{\res}[1]{\Res\displaylimits_{#1}} 
\providecommand{\norm}[1]{\left\lVert#1\right\rVert}
%\providecommand{\norm}[1]{\lVert#1\rVert}
\providecommand{\mtx}[1]{\mathbf{#1}}
\providecommand{\mean}[1]{E\left[ #1 \right]}
\providecommand{\cond}[2]{#1\middle|#2}
\providecommand{\fourier}{\overset{\mathcal{F}}{ \rightleftharpoons}}
\newenvironment{amatrix}[1]{%
  \left(\begin{array}{@{}*{#1}{c}|c@{}}
}{%
  \end{array}\right)
}
\newcommand{\cosec}{\,\text{cosec}\,}
\providecommand{\dec}[2]{\ensuremath{\overset{#1}{\underset{#2}{\gtrless}}}}
\newcommand{\myvec}[1]{\ensuremath{\begin{pmatrix}#1\end{pmatrix}}}
\newcommand{\mydet}[1]{\ensuremath{\begin{vmatrix}#1\end{vmatrix}}}
\newcommand{\myaugvec}[2]{\ensuremath{\begin{amatrix}{#1}#2\end{amatrix}}}
\providecommand{\rank}{\text{rank}}
\providecommand{\pr}[1]{\ensuremath{\Pr\left(#1\right)}}
\providecommand{\qfunc}[1]{\ensuremath{Q\left(#1\right)}}
	\newcommand*{\permcomb}[4][0mu]{{{}^{#3}\mkern#1#2_{#4}}}
\newcommand*{\perm}[1][-3mu]{\permcomb[#1]{P}}
\newcommand*{\comb}[1][-1mu]{\permcomb[#1]{C}}
\providecommand{\qfunc}[1]{\ensuremath{Q\left(#1\right)}}
\providecommand{\gauss}[2]{\mathcal{N}\ensuremath{\left(#1,#2\right)}}
\providecommand{\diff}[2]{\ensuremath{\frac{d{#1}}{d{#2}}}}
\providecommand{\myceil}[1]{\left \lceil #1 \right \rceil }
\newcommand\figref{Fig.~\ref}
\newcommand\tabref{Table~\ref}
\newcommand{\sinc}{\,\text{sinc}\,}
\newcommand{\rect}{\,\text{rect}\,}

\begin{frame}
  \titlepage
\end{frame}


\begin{frame}{Question}
Consider the following regression model
\begin{center}
	$y_{k} = \alpha_{0} + \alpha_{1} \log_{e}k + \epsilon_{k}, \qquad k = 1,2,…,n,$\\
\end{center}
where $\epsilon_{k}$'s are independent and identically distributed random variables each
having probability density function $ f\brak{x} = \frac{1}{2} e^{-|x|}, x \in \mathbb{R}$. Then which one of
the following statements is true? 
\begin{enumerate}
	\item The maximum likelihood estimator of $\alpha_{0}$ does not exist
	\item The maximum likelihood estimator of $\alpha_{1}$ does not exist
	\item The least squares estimator of $\alpha_{0}$ exists and is unique
	\item The least squares estimator of $\alpha_{1}$ exists, but it is not unique
\end{enumerate}
\end{frame}


\begin{frame}[allowframebreaks]{Likelihood Function}
%\begin{itemize}
\begin{align}
	f(\epsilon_{k}) &= \frac{1}{2} e^{-|\epsilon_{k}|}\\
	\text{Likelihood function}: f(\epsilon_{1}\epsilon_{2}....\epsilon_{n}) &= \prod_{k = 1}^{n} f(\epsilon_{k})\\
	L     &= \prod_{k = 1}^{n}\frac{1}{2}e^{-|\epsilon_{k}|}\\
L_{1} = \ln L &= \ln\brak{\prod_{k = 1}^{n}\frac{1}{2}e^{-|\epsilon_{k}|}}\\
	      &= \sum_{k = 1}^{n} \ln\brak{\frac{1}{2}e^{-|\epsilon_{k}|}}
\end{align}
%\end{frame}
%
%
%\begin{frame}
\begin{align}
	      &= \sum_{k = 1}^{n} \brak{-\ln 2 -|y_{k}-\alpha_{0}-\alpha_{1} \log_{e}k|}\\
	      &= -n\ln 2 -\sum_{k = 1}^{n} \brak{|y_{k}-\alpha_{0}-\alpha_{1} \log_{e}k|}\\  
	L_{1} &= \text{function of }\alpha_{0}, \alpha_{1}
\end{align}
We need to find the value of $\alpha_{0}$ and $\alpha_{1}$ which will maximise the value of $L_{1}$ i.e. the value of 
	$\alpha_{0}$ and $\alpha_{1}$ which will minimise the value of $\sum_{k = 1}^{n}|y_{k}-\alpha_{0}-\alpha_{1} \log_{e}k|$
\end{frame}


\begin{frame}[allowframebreaks]{Maximum Likelihood Estimator for $\alpha_{0}$}
\begin{enumerate}
	\item For $y_{k}-\alpha_{0}-\alpha_{1} \log_{e}k > 0$
\begin{mini*}|s|
{\alpha_{0}}{y_{k}-\alpha_{0}-\alpha_{1} \log_{e}k}
{}{}
\addConstraint{\alpha_{0}\leq y_{k}-\alpha_{1}\log_{e}k}{}
\end{mini*}
Using Lagrange multiplier method
\begin{align}
	L\brak{\lambda} &= y_{k}-\alpha_{0}-\alpha_{1} \log_{e}k - \lambda(\alpha_{0}-y_{k}+\alpha_{1} \log_{e}k)\\
	\frac{\partial L}{\partial \alpha_{0}} &= -1-\lambda = 0\\
	\frac{\partial L}{\partial \lambda} &= y_{k}-\alpha_{0}-\alpha_{1} \log_{e}k = 0\\
	\lambda &= -1\\
	\alpha_{0} &= y_{k} - \alpha_{1} \log_{e}k
\end{align}
%\end{enumerate}
%\end{frame}
%
%\begin{frame}
%\begin{enumerate}
\item For $y_{k}-\alpha_{0}-\alpha_{1} \log_{e}k < 0$
\begin{mini*}|s|
{\alpha_{0}}{-\brak{y_{k}-\alpha_{0}-\alpha_{1} \log_{e}k}}
{}{}
\addConstraint{\alpha_{0}\geq y_{k}-\alpha_{1}\log_{e}k}{}
\end{mini*}
Using Lagrange multiplier method
\begin{align}
	L\brak{\lambda} &= -\brak{y_{k}-\alpha_{0}-\alpha_{1} \log_{e}k} - \lambda(\alpha_{0}-y_{k}+\alpha_{1} \log_{e}k)\\
        \frac{\partial L}{\partial \alpha_{0}} &= 1-\lambda = 0\\
        \frac{\partial L}{\partial \lambda} &= y_{k}-\alpha_{0}-\alpha_{1} \log_{e}k = 0\\
        \lambda &= 1\\
        \alpha_{0} &= y_{k} - \alpha_{1} \log_{e}k
\end{align}
As value of $\alpha_{0}$ matches for both cases of modulus\\
$\therefore$ The maximum likelihood estimator of $\alpha_{0}$ exist
\end{enumerate}
\end{frame}


\begin{frame}[allowframebreaks]{Maximum Likelihood Estimator for $\alpha_{1}$}
\begin{enumerate}
	\item For $y_{k}-\alpha_{0}-\alpha_{1} \log_{e}k > 0$
\begin{mini*}|s|
{\alpha_{1}}{y_{k}-\alpha_{0}-\alpha_{1} \log_{e}k}
{}{}
\addConstraint{\alpha_{1}\leq \frac{y_{k}-\alpha_{0}}{\log_{e}k}}{}
\end{mini*}
Using Lagrange multiplier method
\begin{align}
	L\brak{\lambda} &= y_{k}-\alpha_{0}-\alpha_{1} \log_{e}k - \lambda \brak{\alpha_{1}-\frac{y_{k}-\alpha_{0}}{\log_{e}k}}\\
        \frac{\partial L}{\partial \alpha_{1}} &= -\log_{e}k -\lambda = 0\\
	\frac{\partial L}{\partial \lambda} &= -\brak{\alpha_{1}-\frac{y_{k}-\alpha_{0}}{\log_{e}k}} = 0\\
        \lambda &= -\log_{e}k\\
	\alpha_{1} &= \frac{y_{k} - \alpha_{0}}{\log_{e}k}
\end{align}
	\item For $y_{k}-\alpha_{0}-\alpha_{1} \log_{e}k < 0$
\begin{mini*}|s|
{\alpha_{1}}{-\brak{y_{k}-\alpha_{0}-\alpha_{1} \log_{e}k}}
{}{}
\addConstraint{\alpha_{1}\geq \frac{y_{k}-\alpha_{0}}{\log_{e}k}}{}
\end{mini*}
Using Lagrange multiplier method
\begin{align}
	L\brak{\lambda} &= -\brak{y_{k}-\alpha_{0}-\alpha_{1} \log_{e}k} - \lambda \brak{\alpha_{1}-\frac{y_{k}-\alpha_{0}}{\log_{e}k}}\\
        \frac{\partial L}{\partial \alpha_{1}} &= \log_{e}k -\lambda = 0\\
        \frac{\partial L}{\partial \lambda} &= -\brak{\alpha_{1}-\frac{y_{k}-\alpha_{0}}{\log_{e}k}} = 0\\
        \lambda &= \log_{e}k\\
        \alpha_{1} &= \frac{y_{k} - \alpha_{0}}{\log_{e}k}
\end{align}
As value of $\alpha_{1}$ matches for both cases of modulus\\
$\therefore$ The maximum likelihood estimator of $\alpha_{1}$ exist\\
$\therefore$ Option \brak{A} and \brak{B} are incorrect\\
\end{enumerate}
\end{frame}


\begin{frame}[allowframebreaks]{Least Square Estimator}
The least square estimator of $\alpha_{0}$ and $\alpha_{1}$ is $\tilde{\alpha_{0}}$ and $\tilde{\alpha_{1}}$ which will minimise
\begin{align}
Q\brak{\alpha_{0},\alpha_{1}} &=  \sum_{k = 1}^{n} \brak{y_{k}-\alpha_{0}-\alpha_{1} \log_{e}k}^2
\end{align}
\begin{table}[!htb]
	%%%%%%%%%%%%%%%%%%%%%%%%%%%%%%%%%%%%%%%%%%%%%%%%%%%%%%%%%%%%%%%%%%%%%%
%%                                                                  %%
%%  This is the header of a LaTeX2e file exported from Gnumeric.    %%
%%                                                                  %%
%%  This file can be compiled as it stands or included in another   %%
%%  LaTeX document. The table is based on the longtable package so  %%
%%  the longtable options (headers, footers...) can be set in the   %%
%%  preamble section below (see PRAMBLE).                           %%
%%                                                                  %%
%%  To include the file in another, the following two lines must be %%
%%  in the including file:                                          %%
%%        \def\inputGnumericTable{}                                 %%
%%  at the beginning of the file and:                               %%
%%        \input{name-of-this-file.tex}                             %%
%%  where the table is to be placed. Note also that the including   %%
%%  file must use the following packages for the table to be        %%
%%  rendered correctly:                                             %%
%%    \usepackage[latin1]{inputenc}                                 %%
%%    \usepackage{color}                                            %%
%%    \usepackage{array}                                            %%
%%    \usepackage{longtable}                                        %%
%%    \usepackage{calc}                                             %%
%%    \usepackage{multirow}                                         %%
%%    \usepackage{hhline}                                           %%
%%    \usepackage{ifthen}                                           %%
%%  optionally (for landscape tables embedded in another document): %%
%%    \usepackage{lscape}                                           %%
%%                                                                  %%
%%%%%%%%%%%%%%%%%%%%%%%%%%%%%%%%%%%%%%%%%%%%%%%%%%%%%%%%%%%%%%%%%%%%%%



%%  This section checks if we are begin input into another file or  %%
%%  the file will be compiled alone. First use a macro taken from   %%
%%  the TeXbook ex 7.7 (suggestion of Han-Wen Nienhuys).            %%
\def\ifundefined#1{\expandafter\ifx\csname#1\endcsname\relax}


%%  Check for the \def token for inputed files. If it is not        %%
%%  defined, the file will be processed as a standalone and the     %%
%%  preamble will be used.                                          %%
\ifundefined{inputGnumericTable}

%%  We must be able to close or not the document at the end.        %%
	\def\gnumericTableEnd{\end{document}}


%%%%%%%%%%%%%%%%%%%%%%%%%%%%%%%%%%%%%%%%%%%%%%%%%%%%%%%%%%%%%%%%%%%%%%
%%                                                                  %%
%%  This is the PREAMBLE. Change these values to get the right      %%
%%  paper size and other niceties.                                  %%
%%                                                                  %%
%%%%%%%%%%%%%%%%%%%%%%%%%%%%%%%%%%%%%%%%%%%%%%%%%%%%%%%%%%%%%%%%%%%%%%

	\documentclass[12pt%
			  %,landscape%
                    ]{report}
       \usepackage[latin1]{inputenc}
       \usepackage{fullpage}
       \usepackage{color}
       \usepackage{array}
       \usepackage{longtable}
       \usepackage{calc}
       \usepackage{multirow}
       \usepackage{hhline}
       \usepackage{ifthen}
       \newcommand{\myvec}[1]{\ensuremath{\begin{pmatrix}#1\end{pmatrix}}}
%       \usepackage{gvv}
	\begin{document}


%%  End of the preamble for the standalone. The next section is for %%
%%  documents which are included into other LaTeX2e files.          %%
\else

%%  We are not a stand alone document. For a regular table, we will %%
%%  have no preamble and only define the closing to mean nothing.   %%
    \def\gnumericTableEnd{}

%%  If we want landscape mode in an embedded document, comment out  %%
%%  the line above and uncomment the two below. The table will      %%
%%  begin on a new page and run in landscape mode.                %%
%       \def\gnumericTableEnd{\end{landscape}}
%       \begin{landscape}


%%  End of the else clause for this file being \input.              %%
\fi

%%%%%%%%%%%%%%%%%%%%%%%%%%%%%%%%%%%%%%%%%%%%%%%%%%%%%%%%%%%%%%%%%%%%%%
%%                                                                  %%
%%  The rest is the gnumeric table, except for the closing          %%
%%  statement. Changes below will alter the table's appearance.     %%
%%                                                                  %%
%%%%%%%%%%%%%%%%%%%%%%%%%%%%%%%%%%%%%%%%%%%%%%%%%%%%%%%%%%%%%%%%%%%%%%

\providecommand{\gnumericmathit}[1]{#1} 
%%  Uncomment the next line if you would like your numbers to be in %%
%%  italics if they are italizised in the gnumeric table.           %%
%\renewcommand{\gnumericmathit}[1]{\mathit{#1}}
\providecommand{\gnumericPB}[1]%
{\let\gnumericTemp=\\#1\let\\=\gnumericTemp\hspace{0pt}}
 \ifundefined{gnumericTableWidthDefined}
        \newlength{\gnumericTableWidth}
        \newlength{\gnumericTableWidthComplete}
        \newlength{\gnumericMultiRowLength}
        \global\def\gnumericTableWidthDefined{}
 \fi
%% The following setting protects this code from babel shorthands.  %%
 \ifthenelse{\isundefined{\languageshorthands}}{}{\languageshorthands{english}}
%%  The default table format retains the relative column widths of  %%
%%  gnumeric. They can easily be changed to c, r or l. In that case %%
%%  you may want to comment out the next line and uncomment the one %%
%%  thereafter                                                      %%
\providecommand\gnumbox{\makebox[0pt]}
%%\providecommand\gnumbox[1][]{\makebox}

%% to adjust positions in multirow situations                       %%
\setlength{\bigstrutjot}{\jot}
\setlength{\extrarowheight}{\doublerulesep}

%%  The \setlongtables command keeps column widths the same across  %%
%%  pages. Simply comment out next line for varying column widths.  %%
\setlongtables

\setlength\gnumericTableWidth{%
	75pt+%
	75pt+%
	75pt+%
0pt}
\def\gumericNumCols{3}
\setlength\gnumericTableWidthComplete{\gnumericTableWidth+%
         \tabcolsep*\gumericNumCols*2+\arrayrulewidth*\gumericNumCols}
\ifthenelse{\lengthtest{\gnumericTableWidthComplete > \linewidth}}%
         {\def\gnumericScale{1*\ratio{\linewidth-%
                        \tabcolsep*\gumericNumCols*2-%
                        \arrayrulewidth*\gumericNumCols}%
{\gnumericTableWidth}}}%
{\def\gnumericScale{1}}

%%%%%%%%%%%%%%%%%%%%%%%%%%%%%%%%%%%%%%%%%%%%%%%%%%%%%%%%%%%%%%%%%%%%%%
%%                                                                  %%
%% The following are the widths of the various columns. We are      %%
%% defining them here because then they are easier to change.       %%
%% Depending on the cell formats we may use them more than once.    %%
%%                                                                  %%
%%%%%%%%%%%%%%%%%%%%%%%%%%%%%%%%%%%%%%%%%%%%%%%%%%%%%%%%%%%%%%%%%%%%%%

\ifthenelse{\isundefined{\gnumericColA}}{\newlength{\gnumericColA}}{}\settowidth{\gnumericColA}{\begin{tabular}{@{}p{75pt*\gnumericScale}@{}}x\end{tabular}}
\ifthenelse{\isundefined{\gnumericColB}}{\newlength{\gnumericColB}}{}\settowidth{\gnumericColB}{\begin{tabular}{@{}p{75pt*\gnumericScale}@{}}x\end{tabular}}
\ifthenelse{\isundefined{\gnumericColC}}{\newlength{\gnumericColC}}{}\settowidth{\gnumericColC}{\begin{tabular}{@{}p{150pt*\gnumericScale}@{}}x\end{tabular}}

%\begin{longtable}[c]{%
\begin{tabular}[c]{%
	b{\gnumericColA}%
	b{\gnumericColB}%
	b{\gnumericColC}%
	}

%%%%%%%%%%%%%%%%%%%%%%%%%%%%%%%%%%%%%%%%%%%%%%%%%%%%%%%%%%%%%%%%%%%%%%
%%  The longtable options. (Caption, headers... see Goosens, p.124) %%
%	\caption{The Table Caption.}             \\	%
% \hline	% Across the top of the table.
%%  The rest of these options are table rows which are placed on    %%
%%  the first, last or every page. Use \multicolumn if you want.    %%

%%  Header for the first page.                                      %%
%	\multicolumn{3}{c}{The First Header} \\ \hline 
%	\multicolumn{1}{c}{colTag}	%Column 1
%	&\multicolumn{1}{c}{colTag}	%Column 2
%	&\multicolumn{1}{c}{colTag}	\\ \hline %Last column
%	\endfirsthead

%%  The running header definition.                                  %%
%	\hline
%	\multicolumn{3}{l}{\ldots\small\slshape continued} \\ \hline
%	\multicolumn{1}{c}{colTag}	%Column 1
%	&\multicolumn{1}{c}{colTag}	%Column 2
%	&\multicolumn{1}{c}{colTag}	\\ \hline %Last column
%	\endhead

%%  The running footer definition.                                  %%
%	\hline
%	\multicolumn{3}{r}{\small\slshape continued\ldots} \\
%	\endfoot

%%  The ending footer definition.                                   %%
%	\multicolumn{3}{c}{That's all folks} \\ \hline 
%	\endlastfoot
%%%%%%%%%%%%%%%%%%%%%%%%%%%%%%%%%%%%%%%%%%%%%%%%%%%%%%%%%%%%%%%%%%%%%%

\hhline{|-|-|-|}
	 \multicolumn{1}{|p{\gnumericColA}|}%
	{\gnumericPB{\raggedright}\gnumbox[l]{\textbf{parameter}}}
	&\multicolumn{1}{p{\gnumericColB}|}%
	{\gnumericPB{\raggedright}\gnumbox[l]{\textbf{value}}}
	&\multicolumn{1}{p{\gnumericColC}|}%
	{\gnumericPB{\raggedright}\gnumbox[l]{\textbf{description}}}
\\  
\hhline{|---|}
	 \multicolumn{1}{|p{\gnumericColA}|}%
	{\gnumericPB{\raggedright}\gnumbox[l]{$n$}}
	&\multicolumn{1}{p{\gnumericColB}|}%
	{\gnumericPB{\raggedright}\gnumbox[l]{10}}
	&\multicolumn{1}{p{\gnumericColC}|}%
	{\gnumericPB{\raggedright}\gnumbox[l]{Number of bulbs in the bag}}
\\
\hhline{|---|}
	 \multicolumn{1}{|p{\gnumericColA}|}%
	{\gnumericPB{\raggedright}\gnumbox[l]{$p$}}
	&\multicolumn{1}{p{\gnumericColB}|}%
	{\gnumericPB{\raggedright}\gnumbox[l]{$\frac{1}{50}$}}
	&\multicolumn{1}{p{\gnumericColC}|}%
	{\gnumericPB{\raggedright}\gnumbox[l]{Bulb chosen is defective}}
\\
\hhline{|---|}
	 \multicolumn{1}{|p{\gnumericColA}|}%
	{\gnumericPB{\raggedright}\gnumbox[l]{$q$}}
	&\multicolumn{1}{p{\gnumericColB}|}%
	{\gnumericPB{\raggedright}\gnumbox[l]{$\frac{49}{50}$}}
	&\multicolumn{1}{p{\gnumericColC}|}%
	{\gnumericPB{\raggedright}\gnumbox[l]{Bulb chosen is proper}}
\\
\hhline{|---|}
	 \multicolumn{1}{|p{\gnumericColA}|}%
	{\gnumericPB{\raggedright}\gnumbox[l]{$\mu$}}
	&\multicolumn{1}{p{\gnumericColB}|}%
	{\gnumericPB{\raggedright}\gnumbox[l]{$\frac{1}{5}$}}
	&\multicolumn{1}{p{\gnumericColC}|}%
	{\gnumericPB{\raggedright}\gnumbox[l]{Mean of the distribution}}
\\
\hhline{|---|}
	 \multicolumn{1}{|p{\gnumericColA}|}%
	{\gnumericPB{\raggedright}\gnumbox[l]{${\sigma}^2$}}
	&\multicolumn{1}{p{\gnumericColB}|}%
	{\gnumericPB{\raggedright}\gnumbox[l]{$\frac{49}{250}$}}
	&\multicolumn{1}{p{\gnumericColC}|}%
	{\gnumericPB{\raggedright}\gnumbox[l]{Variance of the distribution}}
\\
\hhline{|-|-|-|}
%\end{longtable}
\end{tabular}

\ifthenelse{\isundefined{\languageshorthands}}{}{\languageshorthands{\languagename}}
\gnumericTableEnd

	\caption{Variables used}
	\label{table_gate23_st_26}
\end{table}
\begin{align}
	%Q\brak{\alpha_{0},\alpha_{1}} &=  \sum_{k = 1}^{n} \brak{y_{k}-\alpha_{0}-\alpha_{1} \log_{e}k}^2\\
\frac{\partial Q}{\partial \alpha_{0}} &= -2 \sum_{k = 1}^{n} \brak{y_{k}-\alpha_{0}-\alpha_{1} \log_{e}k} = 0\\
	\sum_{k = 1}^{n} \brak{y_{k}-\alpha_{0}-\alpha_{1} \log_{e}k} &= 0\\
	n\bar{y} - n\alpha_{0} - \alpha_{1}n\bar{x} &= 0 \\%,\quad \text{where } \bar{x} = \frac{1}{n} \sum_{k = 1}^{n}\log_{e}k,\quad \bar{y} = \frac{1}{n} \sum_{k = 1}^{n}y_{k}\\
	\implies \tilde{\alpha_{0}} &= \bar{y} - \tilde{\alpha_{1}}\bar{x}\\
	\frac{\partial Q}{\partial \alpha_{1}} &= -2 \sum_{k = 1}^{n} \brak{y_{k}-\alpha_{0}-\alpha_{1} \log_{e}k}\log_{e}k = 0\\
	\implies \tilde{\alpha_{1}} &= \frac{\sum_{k = 1}^{n}\brak{\log_{e}k-\bar{x}}\brak{y_{k}-\bar{y}}}{\sum_{k = 1}^{n}\brak{\log_{e}k-\bar{x}}^2}
\end{align}
$\therefore$ Least square estimator of $\alpha_{0}$ and $\alpha_{1}$ exists and are unique\\
$\therefore$ Option \brak{C} is correct and \brak{D} is incorrect\\
\end{frame}


\begin{frame}[allowframebreaks]{Simulation}
probability density function is $ f\brak{x} = \frac{1}{2} e^{-|x|}$
\begin{enumerate}
\item Write a function cdf for calculating the cdf of any random variable\\
	 \begin{align}
  p_X(x) &= 
  \begin{cases}
          \frac{1}{2} e^{x} & x \le 0
  \\
          \frac{1}{2} e^{-x} &  x > 0
  \end{cases}
  \end{align}

  \begin{align}
  F_X(x) &= 
  \begin{cases}
	  \int_{-\infty}^{x} \brak{\frac{1}{2} e^{x}} dx  & x \le 0
  \\
	  \int_{-\infty}^{0} \brak{\frac{1}{2} e^{x}} dx + \int_{0}^{x} \brak{\frac{1}{2} e^{-x}} dx &  x > 0
  \end{cases}
  \end{align}

  \begin{align}
  F_X(x) &= 
  \begin{cases}
	  \frac{1}{2} e^{x} & x \le 0
  \\
	  \frac{1}{2} \brak{2-e^{-x}} &  x > 0
  \end{cases}
  \end{align}
\item Declare a function inverse cdf \brak{I\brak{u}} such that its input is any random number 
	and output is random variable whose cdf equals that of the given distribution\\
	For x $\le$ 0
		\begin{align}
			u &= \frac{1}{2} e^{x}\\
			e^{x} &= 2u\\
			x &= \ln{2u}\\
			\because x \le 0\\
			u \le 0.5
		\end{align}
	For x $>$ 0
		\begin{align}
			u &= \frac{1}{2} \brak{2-e^{-x}}\\
			2-e^{-x} &= 2u\\
			e^{-x} &= 2-2u\\
			x &= -\ln \brak{2-2u}\\
			\because x > 0\\
			u > 0.5
		\end{align}
  \begin{align}
  I\brak{u} &=
  \begin{cases}
	  \ln\brak{2u} & u \le 0.5
  \\
	  -\ln \brak{2-2u} &  u > 0.5
  \end{cases}
  \end{align}
\item Define three arrays random\textunderscore{vars} , cdf\textunderscore{values}
	, theoretical\textunderscore{cdf}\textunderscore{values}
	to store random variables, simulated cdf values and theoretical cdf values
\item Generate random numbers using rand() and calling inverse cdf funtion to generate our random variable
\item Calling cdf function to calculate the cdf of the generated random variable
\item Storing the random variable,theoretical cdf and generated cdf into their respective arrays
\item Storing the data of these three array into a .dat file
\item Plotting these .dat file in python
\end{enumerate}
\end{frame}
\end{document}

